\chapter{Blade Vortex Interactions}
The Blade Vortex Interaction (BVI) is one of the most complicated pheomena appearing in the operational environment of the rotating wing aircrafts. Under certain conditions, tipically in fast forward flight and high load manouvering, the vortical structure rising from the tip of a rotor blade can intersect the other blades, thus modifying their aerodynamic behaviour. The main consequences of these interactions can be either large loads fluctuations, large noise and in extreme cases the trigger of the dynamic stall\scitec{Caradonna84}\scitec{Zhao17}\scitec{Martin86}\scitec{Johnson89}\scitec{Lorber93}\scitel{Chaderjian17}. Before going through the discussion about BVIs it is useful to analyze the phenomenology of the main character of our problem: the tip vortex.

\section{Vortex}

At the beginning of his studies on turbulent flows \textcite{Robinson90} proposed a single sentence to clarify the concept of vortex:\\

\textit{“A vortex exists when instantaneous streamlines mapped onto a plain normal to the vortex core exhibit a roughly circular or spiral pattern, when viewed from a reference frame moving with the center of the vortex core.”}\\

However he clearly pointed out the problem of knowing a-priori the position, the orientation and the velocity of the vortex reference frame,  which it is undoubtely not an easy problem to solve. Indeed it is surprisingly hard to formulate a clear and unique definition of \textit{vortex}. Usually many different features of this physical phenomenon are used to outline it\scitel{Bauknecht16}:

\begin{itemize}
	\item a region with circular flow,
	\item a region of high vorticity,
	\item a region with reduced static pressure,
	\item a region surrounded by closed or spiral streamlines.
\end{itemize}

However there are many cases of flow, in which the cited features are present but no vortex is visible. Nevertheless, many criteria have been succesfully developed to identify the presence of a vortex starting from the flow field characteristics\scitel{Haller05}. These criteria are useful both in numerical and experimental field in order to localize the core center, its dimensions and the magnitude of vortices (some of them will be explained and used in this work in the data analysis section). 

Some important parameters concerning vortices must be introduced. Every vortex (apart from theoretical limit cases) owns a certain amount of vorticity:

\begin{equation}
	\gls{omega} = \nabla \times \gls{u}
\end{equation}

Vorticity is the curl of the velocity field, and usually is mainly bounded in the core region. The circulation is a measure of the vortex intensity. It is defined as the line integral of the velocity field:

\begin{equation}
	\gls{Gamma} = \oint_{\mathcal{C_0}} \gls{u} \cdot \bm{\hat{\tau}} d\ell
\end{equation}

It can be easily related to vorticity using Stokes' Theorem:

\begin{equation}
	\gls{Gamma} = \oint_{\mathcal{C_0}} \gls{u} \cdot \bm{\hat{\tau}} d\ell = \int_{\mathcal{S_0}} \nabla \times \gls{u} \cdot \bm{\hat{n}} d\mathcal{S} = \int_{\mathcal{S_0}} \gls{omega} \cdot \bm{\hat{n}} d\mathcal{S}
\end{equation}

Circulation calculated on a line surrounding the core region, at least theoretically, should be independent from the choice of the line itself, in other words it is constant: in the reality it is more likely an asyntotical behaviour. The farther from the center this calculation is performed, the closer to the total vortex circulation the result is expected to be.

%VALUTARE
%Once formed, the tip vortices induce a velocity according to the Biot-Savart law, hence they induce a negative velocity on the wing giving rise to the so called induced drag.
%VALUTARE


\subsection{Fixed Wing Tip Vortices}

In nature there are several situations in which a vortex like motion appears (whirpools, tornadoes...), however this work is focused on vortices generated by a three dimensional finite span wing\scitec{Green06}\scitel{Anderson17}. 

It is well known that finite span wings in lifting condition generate two counter rotating vortices at their tips: many smoke flow visualizations have been performed in years in order to characterize this motion. 
%Curiously these vortices can appear as well behind the rear wing of the F1\textsuperscript{\textregistered} cars during races in particularly humid days when the moisture present in the air is forced to condensate at vortex center.% (Figure \ref{F1_tip}). 

%\begin{figure}
%	\centering
%	\includegraphics[scale=0.3]{img/Canc_F1.jpg}
%	\caption{Example of tip vortices on F1 car}
%	\label{F1_tip}
%\end{figure}

Usually this phenomenon can be explained in three ways. Starting from the most intuitive one: the pressure difference between top and bottom of the wing leads the air to curl at the wing tip: the air on the lower surface of the airfoil is \textit{sucked} towards the top, thus generating a swirling motion. Therefore, on both sides of the wing is present a spanwise velocity component: towards the tip on the bottom, while towards the midchord on the top (Figure \ref{Finite_Wing}). The shear layer rising at the trailing edge, due to the different direction of the streamlines, is then progressively rolled up into the tip vortices once it has left the profile. 

\begin{figure}
	\centering
	\includegraphics[scale=0.8]{img/Canc_VTip.png}
	\caption{Streamlines on a finite span wing}
	\label{Finite_Wing}
\end{figure}

Another theory involves the shear layer generating at the tip between the undisturbed flow passing just aside the wing and the perturbed flow passing over the wing. In this case, vorticity arises for the non parallelism of the streamlines in this region, leading to the rise of the tip vortex. 

The most theoretical explanation, instead, involves Kelvin's and Helmholtz's theorems regarding circulation and vortex filaments: under hypothesis of incompressible fluid and negligible viscosity, the circulation along a closed line moving with the fluid is constant; moreover, under the same hypothesis it can be demonstrated that a vortex filament can not end by itself in the fluid but has to close in a toroidal structure. The wing can be modelled as a lifting vortex oriented perpendicularly to the freestream velocity. In this view, tip vortices are the branches conjuncting the lifting vortex of the wing to the starting vortex generating when the asintotical velocity suddenly changes from zero to a finite positive value at initial time. According to the lifting line theory, moreover, the vorticity shed in the wake is proportional to the variation of the circulation along the wing span\scitel{Quartapelle13}.

Several experimental studies have featured the different topological regions and the development in time of the fixed wing tip vortices, analyzing their dependance upon flow and geometric parameters. The rise of the vortex takes place approximately at a quarter of the chord, where the air from the lower surface reaches the top of the wing during its curling motion. This bulk of fluid constitutes the inner part of the vortex core: usually its diameter is in the order of 2\% of the wing chord. At the same time a second flow coming from the inner part of the upper surface comes into play by surrounding the former region. At the beginning of their life, these two flows are turbulent: due to their high rotational speed, however, the motion is quickly stabilized to an almost laminar rigid rotation. The tangential velocity in this region is linear with the radial coordinate $\gls{r}$, though it can be different for the two concentric rings. The separation of the two laminar regions dies out quickly after the vortex  generation: the inner part progressively expands to reach the core boundary whose diameter is usually about 5-10\% of the chord\scitel{Devenport96}. Initially there is also a strong component of axial velocity in the center of the vortex core, due to the natural low pressure region in the high rotation zone. Usually this velocity settles rapidly down to values close to the freestream velocity just after the vortex leaves the trailing edge.

The concept of vortex core, containing these laminar regions, is usually defined through its diameter, supposing it has an approximately circular shape: the core diameter is the distance between two points which have both the maximum tangential velocity and are located at opposite sides of the core. Outside this laminar region, the flow starts to turn to turbulent and, after a short transition region, the tangential velocity is no more linear but varies almost as $1/\gls{r}$ where $\gls{r}$ is the radial coordinate.

The tip vortices start to roll up the wake just after they have left the trailing edge. The consequence is a loss of the initial symmetry; however, the shape of the vortices, if scaled with the right length, shows an auto similar behaviour in time. Indeed, the elapsed time from the formation of the vortex (its age), together with the wing lift coefficient, are the two main parameters which regulate the tip vortex features. In the first stages of vortex life it is registered a slight increase of the core dimension, followed by a logaritmic decrease as a function of the age. A similar behaviour is registered for the tangential velocity. The axial velocity, after its settling, is affected by the wake roll up and increase progressively with the wortex age\scitel{Rorke73}.

Later on the fluid viscosity will slowly dissipate the vortex, increasing the core diameter and decreasing its intensity (circulation). Moreover, a typical instability that has been observed in experimental studies might occurr: smooth sinusoidal wavetypes are low frequencies motions that lead the vortex to move in an oscillatory manner with respect to its original axis. This instability can quicky cause the vortex breakdown.

Particular attention must be paid while performing an experimental vortex characterization to the phenomenon of \textit{wandering}. As the name suggests, it consists of small movements of the vortex core from one run to the other. The causes of this discrepancy are not yet well understood, however many authors address it to the fluctuations of velocity inside the wind tunnel. The real problem of wandering is that in some cases it can also force a rotation of the vortex axis, actually changing the values of the acquisitions: this is particularly problematic when the movements of the core correspond approximately to the core diameter, so data can not be clearly and immediately discarded. 

\begin{figure}
	\centering
	\includegraphics[scale=2]{img/vortex_vis.png}
	\caption{Different regions of wing tip vortex visualized by \textcite{Johnson10}: 1) Laminar inner core 2) Transition region 3) Fully turbulent region}
	\label{vortex_vis}
\end{figure}


%\begin{table}
%	\centering
%	\caption{Features of different Tip Vortices for untwisted wing cases}
%	\label{tip_v_feat}
%	\begin{tabular}{cSS}
%		\toprule
%		\centering
%	                     & {Fixed Wing} & {Rotary Wing} \\
%		\midrule
%		$\Gamma$         &	      &    \\
%		$d_{core}$       &         &     \\
		
%		\bottomrule
%	\end{tabular}
%\end{table}

\subsection{Rotary Wing Tip Vortices}

Tip vortices and wake formation show some differences between fixed wings and helicopter blades. 

As pointed out by the lifting line theory, the vorticity per unit span shed in the wake is proportional to the variation of the bound circulation on the wing. On a fixed wing, the maximum value of circulation is usually registered at midspan, and it progressively decreases while going towards the wing tip\footnote{Untwisted wings are considered for the comparison.}. Thus, at least for a semi span, the wake vorticity has always the same direction, driving the roll up of the wake on the tip vortex. 

The circulation distribution over an helicopter blade is foundamentally different, due to its particular motion which implies lower Reynolds and Mach numbers at the root than at the tip. This leads to a triangular $\gls{Gamma}$ distribution having its maximum around 85-95\% of the blade radius. This means that, at least theoretically, the vorticity present in the tip vortex should be equal to the peak of the bound vorticity and should be of opposite direction with respect to the inner wake (Figure \ref{CRV}). In fact some wake models account for the presence of an inner weaker counter rotating vortex which marks the border of the inner vortex sheet. In this case the blade tip vortex is less influenced by the presence of the remaining part of the wake, since no roll up is present. 

However recent studies\scitel{Komerath04} have highlighted how the intensity of the rotor tip vortices is roughly only 40\% of the peak of maximum bound circulation; this is probably a consequence of a fluid exchange (in the first stages of the wake life) between the outer vortex and the counter rotating vortex which, in turn, is proved to hold a larger $\gls{Gamma}$ value than expected. After this reduction of circulation, the vortex, if not intercepting any blade, keeps his core dimension unchanged for several blades rotations. The shape of a rotor blade tip vortex seems to be more elliptical rather than circular, wider in the horizontal direction while slimmer in the vertical.


\begin{figure}
	\centering
	\includegraphics[scale=1]{img/CRV.png}
	\caption{Different circulation distribution for fixed wing and helicopter blade\scitel{Komerath04}}
	\label{CRV}
\end{figure}


\section{BVI Classification and Identification}

\begin{figure}
	\centering
	\includegraphics[scale=0.9]{img/vortex_directions.png}
	\caption{Difference between Parallel, Perpendicular and Oblique BVI}
	\label{vdirection}
\end{figure}

Usually it is common to divide BVIs in three (possibly four) categories (Figure \ref{vdirection}): 

\begin{itemize}

	\item{\textbf{Perpendicular BVI}} : it happens when the axis of the vortex is perpendicular to the blade. This type of interaction is usually strongly three dimensional since the induced velocity coming from the vortex structure clearly varies as a function of the blade span. Even though the perpendicular BVI usually involves just a small portion of the blade span, it is recognised to be a possible cause of the triggering of the dynaic stall\scitel{Richez17}. In fact, at least on one side of the vortex, the induced velocity contributes to increase drastically the angle of attack of the blade, and easily leads the profile into a stall zone. Being this phenomenon steady for almost half of the rotor disk, the noise generated is mostly a broadband noise\scitel{Gibertini14}.
	
	\item{\textbf{Parallel BVI}} : it takes place when the axis of the vortex is parallel to the blade. This interaction is an unsteady phenomenon: a single blade could experience the encounter with a parallel vortex many times during a single rotation. A parallel BVI usually involves larger parts of the blade span if compared to the former, and for this reason it is considered (and studied too) as a two-dimensional phenomenon. In fact there is no need to consider any components of the velocity perpendicular to the airfoil section. The noise generated by a parallel BVI is the most intense: with its large pressure fluctuations close to the blade leading edge, this interaction is the most critical for the detectability of the machine\scitel{Yu00}. 
	
	\item{\textbf{Oblique BVI}} : this classification identifies those interactions which show characteristics from both the previous depicted BVIs.
	
	\item{\textbf{Orthogonal BVI}} : this last classification includes basically all the interactions of the blade tip votices with the tail rotor. These BVIs take place when the vortex axis and the plane of the rotor disk are orthogonal\scitel{Green06}.

\end{itemize}

\begin{figure}
	\centering
	\subfloat[$\gls{mu} = 0.13$]{\includegraphics[scale=0.5]{img/epiwake013.eps}}
	\subfloat[$\gls{mu} = 0.30$]{\includegraphics[scale=0.5]{img/epiwake030.eps}}
	\caption{Epicycloidal wake model for 4 blades rotor}
	\label{epiwake}
\end{figure}

A primary task for researchers has been (and still is) the identification of the location where the BVIs are more likely to happen. It is outlined in many experimental studies that the parallel BVI is usually registered in the first and in the fourth quadrant of the rotor disk. Interactions in the first quadrant are usually less effective on aerodynamic loads in terms of pressure coefficient, but at high velocities they can lead to the formation of shock waves\scitel{Nakamura81}. On the other hand the interactions in the fourth quadrant cause higher loads fluctiations (and higher noise) and in some cases can lead the profile to a dynamic stall condition. Perpendicular BVI, instead, takes place mainly in the second and third quadrant and, expecially in this last location, it could play a negative role in the separation of the flow on the upper side of the profile.

Simulation tools have been developed in years in order to try to foresee BVI locations: the key role in this field is played by the wake geometry. 
A very simple approach to the problem is to consider the tip vortices to form an epicycloidal wake (Figure \ref{epiwake}) that convects downstream with the same airstream velocity. Results in terms of location inside the rotor disk are quite good, however, no information about the miss distance can be obtained from this analisys. Formulations of the problem like Beddoes or Mangler-Squire\scitel{Anderson17} (Figure \ref{MSwake}) can provide this degree of freedom to the wake allowing it to travel also in the direction perpendicular to the rotor disk of a quantity calculated using the mean induced velocity. These models, in which the wake geometry is given by the user and fits to the flight condition with many empirical coefficients, are called \textit{prescribed wake models}: they have a really low computational cost but their validity is mostly restricted to specifical flight conditions or rotor types. 

\begin{figure}
	\centering
	\subfloat[$\gls{mu} = 0.13$]{\includegraphics[scale=0.5]{img/MS013.eps}}
	\subfloat[$\gls{mu} = 0.30$]{\includegraphics[scale=0.5]{img/MS030.eps}}
	\caption{Mangler - Squire wake model for 4 blades rotor}
	\label{MSwake}
\end{figure}


On the other hand, the so called \textit{free wake models} calculate the wake geometry as a part of the solution, obtaining a more precise description of the rotorcraft environment. These codes are mainly based on a potential flow model, using a lifting line or lifting surface to describe the blades. The wake is usually represented by vortex sheets or vortex filaments, and in more recent developments it can also be transformed into vortex particles (Figure \ref{wake_vis} - \ref{free_wake_interactions}). This approach is Lagrangian, in fact the vorticity of the wake is governed by equations describing its transport and evolution in time. The more realistic representation of the wake has not only pros, but also cons: in this case the computational cost increases significantly\scitec{Scully75}\scitel{VanHoydonck09}.


\begin{figure}
	\centering
	\includegraphics[scale=0.3]{img/wake_vis.eps}
	\caption{Example of free wake model: vortex particles}
	\label{wake_vis}
\end{figure}

\begin{figure}
	\centering
	\subfloat[]{\includegraphics[scale=0.5]{img/perpendicular_wake013.eps}}
	\subfloat[]{\includegraphics[scale=0.5]{img/parallel_wake013.eps}}
	\caption{Perpendicular and parallel interaction calculated from the free wake model}
	\label{free_wake_interactions}
\end{figure}