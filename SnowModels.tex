% ---------------------------------------------------------------- %
% !TEX encoding = UTF-8 Unicode
% !TEX TS-program = pdflatex
% !TEX root = main.tex
% !TEX spellcheck = en-EN
% ---------------------------------------------------------------- %

\chapter{Review of Falling-Snow Models}
	In this chapter the theoretical framework behind a general drag coefficient model will be explained and a review of the available models in literature will be proposed. 
	
	Recalling from \ref{sec: NonSphericalParticles}, the formula used for these models is the following:
	\begin{equation}
		c_D = c_D(\dv, Re, \underline{\varPhi})
	\end{equation}
	
	The characteristic dimension of the particle ($ \dv $) 
	
	
	Generic non-spherical particle model description.\\
	formula (recalled from intro)\\
	Principal shape parameters

	\section{Chhabra review}
		The models must provide information on 2 main aspect: \textit{shape} of the particle (sphericity) and \textit{orientation} of the particle. (Chhabra)
		
	\section{Ganser - 1993}	
		Description of the Ganser model (The best up to 1993).
				
	\section{Heymsfield and Westbrook - (2004)}
		Description of the model ans why it doesn't work.
		
	\section{Holzer and Sommerfeld - (2008)}
		Description of the model and it's simplified form. Comparison with Ganser: sensitivity study on the parameters of both models and why H\&S better suits the experimental curves of snow
		(+ "cite" the ICE GENESIS results proving that this model is the better one)
		
	\section{Model comparison}
		comparison between the models and justification of my choice.
	
	\section{Terminal Velocity calculation}
		Equation for the terminal velocity of a particle: how to calculate every term starting from the diameter and the shape parameters
	
